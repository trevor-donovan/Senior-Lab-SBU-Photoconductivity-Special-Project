\documentclass{article}
\usepackage{graphicx} % Required for inserting images

\title{PC Lab Measurements + Analysis, First Draft}
\author{Trevor Donovan}
\date{}

\begin{document}

\maketitle

\section*{Measurements}

You will have about 8 3-hour class sessions over a period lasting a month to take measurements, do analysis, and hand in your paper. This experiment demands a number of long, autonomous measurements, so you may only need to be present at the beginning of most class periods, but it is your choice. Use that extra time to write the necessary code to do analysis on data as it is acquired.

The primary goal of this experiment is to measure the photoconductivity of a few sample materials. A supplementary goal of this experiment is to measure the dependence of conductivity or resistivity on quantities such as temperature, magnetic field, and previous sample state (hysteresis).

We will use the VersaLab to provide the precise temperature and field control, and laser light coupled to into the probe to provide the light that causes photoconductivity. Unfortunately, there is a key issue: the laser heats the sample, making it more conductive. The increase in conductivity from an incident laser is thus not entirely photoconductivity, since increasing temperature also normally increases conductivity in semiconductors. This would be no issue if the chamber was accurately measuring the temperature of the sample itself, but data taken during the designing of the experiment suggests otherwise (symmetrically-illuminated samples should display very little photoconductivity, but we did so and conductivity decreased a decent amount). As such, separating the thermal and photo effects may be tricky.

Before beginning the measurements sequence, make sure the pressure is low (around a few Torr). For laser measurements, make sure the laser is actually on before measurements begin and will not shut off. The fault locator laser has a battery life of about 10 hours(long enough for a long measurement, but not two or three), so it may be necessary to replace it before each long measurement.

When writing sequences, make sure to adhere to the following guidelines. Since the setup will be sitting for a while at the end of the sequence before anyone touches it again, it is ideal to set the temperature to 300 or 298 Kelvin at the end of the sequence. To measure the expected temperature hysteresis and compare to materials without it, the sequence must sweep up and down in temperature. Keep track of which direction (hot to cold or cold to hot) happens first, as you will want to separate these sections in the analysis. Note that though you are measuring resistances, MultiVu calls the measurement resistivity because it is made with an assumed sample shape in mind. Our samples are in general less regularly shaped, so make sure to select Ohms as the unit of measurement.

If the voltage, power, or current limits are met during the course of a measurement, split the sequence into two (or four, if it goes in both directions in temperature) temperature sweeps with appropriately adjusted voltage, power, and current limits. This (in addition to broken sample wiring) is a likely solution to any flat sections of graph that occur.

A suggested measurement schedule is given below, designed to require 6 class periods of measurement. More time may be necessary in later class periods or outside of class, or one of the three sample materials can be skipped if time requires.

\begin{itemize}
    \item \textbf{Class 1:} Familiarize yourselves with using the VersaLab and MultiVu safely and as intended, using the TA, instructor, and manuals as reference resources whenever something about the setup is unsure. First, test out a single resistance measurement of a sample of known resistance to verify which channels to use. Then, write a practice sequence to measure this repeatedly and save the data to a file, sweeping over small temperature and magnetic field ranges (keeping in mind that the chamber temperature will actually change much slower than the rate given in the sequence), and check that the data file is produced correctly. Finally, if everything has gone correctly, replace the known sample with the vanadium dioxide sample, then write and start a sequence that sweeps temperature from 200-400 Kelvin and back, also sweeping through a small number of magnetic field values at each temperature. Make sure the temperature points are more than finely spaced enough to measure the expected drop or rise in resistivity, but try to keep the expected duration less than 10 hours.
    \item \textbf{Class 2:} After securing the data from the previous class's acquisition, repeat class 1's measurement but with a bright laser fiber-coupled to the probe. In order to create the proper bias of charge carriers, make sure the light is not shining uniformly on the sample.
    \item \textbf{Class 3-6:} Repeat the above, but for the two other samples (gallium arsenide and indium arsenide)
\end{itemize}

After a given measurement is done, the data will be saved to the file specified in the sequence. Copy the .dat files (e.g. by using a portable hard drive) to your computers. Each data file contains a header by default, which matters because it is preferable to convert the .dat files to .csv files by simply changing the file handle of the data files. Before converting the file type, you must delete this header so the top row of the .csv file is made of strings and each line below is made of numbers. After you have these .csv files(which can be checked for correctness in a program such as Jupyter Notebook), you are ready to analyze the data.

\section*{Analysis}

Assuming analysis is done in Python: It is recommended that (for each student) the analysis be done in a program such as Jupyter Notebook or Visual Studio Code with many sequential blocks of code in one file. Modules such as pandas (by converting .csv files to the malleable dataframe object type and making use of their many sorting and selecting functions), Numpy (for handling any arrays of numbers in typically more familiar ways than pandas), SciPy (for fitting curves to data and doing all statistics), and Matplotlib (for visualizing data) will prove useful for analysis.

Whenever you acquire a curve of resistance, plot it first. Given the multiple variables at play, think carefully about how and what to plot to most transparently communicate the important details of the data. Fit curves to the data to determine graph statistics.

Consider how to isolate the photoconductivity from the thermal conductivity, considering that the temperature value given is not the temperature of the upper surface of the sample when the sample is left bathing in laser light. Can you theoretically estimate the temperature increase of the sample due to the incident light?

It is resistance or conductance that is directly measured, but we are interested in resistivity or conductivity. Can you model the shape of the sample to determine conductivities from the measured resistances? Can you estimate the carrier number and carrier density due to temperature, magnetic field, and photoconductivity?

\end{document}
